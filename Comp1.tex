\documentclass[12pt]{article}

% Language setting
\usepackage[english]{babel}

% Set page size and margins

\usepackage[letterpaper,top=2cm,bottom=2cm,left=3cm,right=3cm,marginparwidth=1.75cm]{geometry}

% Useful packages
\usepackage{amsmath}
\usepackage{graphicx}
\usepackage[colorlinks=true, allcolors=blue]{hyperref}
\usepackage{listings}
\lstset {
language=C++,
backgroundcolor=\color{yellow!20},
commentstyle=\color{green},
%keywordstyle=\color{blue},
basicstyle=\footnotesize
}

\usepackage{xcolor}
\usepackage{hyperref}
\hypersetup{
    colorlinks=true,
    linkcolor=black,
    filecolor=black,
    urlcolor=black,
    citecolor=black}

\usepackage{graphicx}
\graphicspath{{images/}}

\title{Compulsory 1}
\author{Rebekka Oskarsen}
\date{31. Januar 2024}
\begin{document}
\maketitle
\tableofcontents
\newpage

\section{Introduction}
In this document I am doing task 1, which I need to have a function and the derivative of that function and definition quantity. Have a text file that shoes the data points and a window that shows the graph of the function.
In task 2 I need to make spiral and have a text file with data points.
Task 3 I need to choose a function with two variables and choose a area in the xy plane, make text file with xyz linearly.

\section{Task 1}
\subsection{Information gathering}
\begin{enumerate}
  \item Find a function:\[ f(x)=cos^2(x)+sin^3(x) \]
  \begin{figure}[htp]
\centering
\includegraphics[width=0.50\linewidth]{Screenshot 2024-01-30 110157.png}
\caption{\label{fig:Graph}This is the GeoGebra graph that shows how my function looks like in a graph.}
\end{figure}

Choose a definition quantity: \[ Df = [-1, 1] \]

    \item Intervals: \[ h = \Delta x = \frac{b-a}{n} \] 
Using this one that I had to rearrange:
\[n= \frac{h}{b-a}\]
Because this formula will help to find number of intervals n given a,b, and desired resolution h.
\end{enumerate}
  \newpage
  \subsection{C++ code}
 function and the derivative of \( f(x)=cos^2(x)+sin^3(x) \)
\begin{figure}[htp]
\centering
\includegraphics[width=0.50\linewidth]{Screenshot 2024-01-30 102611.png}
\caption{\label{fig:c++code} }
\end{figure}

\begin{lstlisting}[language=C++, caption={Function and Derivative}]
float function(float x) 
{
    return pow(cos(x), 2) + pow(sin(x), 3);
}

float calculateDerivative(float x, float h) 
{
    return (function(x + h) - function(x)) / h;
}
\end{lstlisting}

\begin{lstlisting}[language=C++, caption={draws the graph}]
glDrawArrays(GL_LINE_STRIP, 0, vertices.size() / 3);
\end{lstlisting}
This line of code is in the render loop, it is used to draw the graph using the line strip.

\begin{figure}[htp]
\centering
\includegraphics[width=0.5\linewidth]{Screenshot 2024-01-30 182926.png}
\caption{\label{fig:c++ Graph}  }
\end{figure}
\begin{lstlisting}[language=C++, caption={Positiv green, negativ red}]
const char* vertexShaderSource = 
	"#version 330 core\n"
	"layout (location = 0) in vec3 aPos;\n"
	"out vec4 color;\n"
	"void main() {\n"
	"   gl_Position = vec4(aPos.x, aPos.y, 0.0f, 1.0f);\n"
	"   color = vec4(aPos.z > 0.0f ? 0.0f : 1.0f, 
            aPos.z > 0.0f ? 1.0f : 0.0f, 0.0f, 1.0f);\n"
	"}\0";
\end{lstlisting}
It also sets the color based on the sign of the z coordinate (aPos.z). If aPos.z is positive, the color is set to green, if not positive then sets to red.

What i actually wanted to show in the window, what i wanted to render is this:
\begin{figure}[htp]
\centering
\includegraphics[width=0.5\linewidth]{Screenshot 2024-01-31 055416.png}
\caption{\label{fig:GeoGebra Graph} GeoGebra }
\end{figure}

\subsection{Text file}
\begin{figure}[htp]
\centering
\includegraphics[width=0.5\linewidth]{Screenshot 2024-01-31 051320.png}
\caption{\label{fig:c++ data-points.txt} }
\end{figure}



\newpage
\section{Task 2}
\begin{lstlisting}[language=C++, caption={draws the spiral}]
glDrawArrays(GL_LINE_STRIP, 0, numPoints);
\end{lstlisting}
This line of code that is in the loop, renders the spiral.

\begin{figure}[htp]
\centering
\includegraphics[width=0.5\linewidth]{Screenshot 2024-01-30 223300.png}
\caption{\label{fig:Spiral}  }
\end{figure}
As you can see i didn't get a spiral. I got a half of a circle or the issue is that i am not rendering in 3D so am not seeing it in a right angle.

\newpage
\subsection{Text file}
\begin{figure}[htp]
\centering
\includegraphics[width=0.4\linewidth]{Screenshot 2024-01-31 062003.png}
\caption{\label{fig:Spiral_data}  }
\end{figure}

\newpage
\section{Task 3}
I choose as my function with two variables:\( f(x)=cos^2(x)+sin^3(x) \)

\begin{lstlisting}[language=C++, caption={Function with two variables}]
//function to calculate the function f(x,y) = x*y^2
float f(float x, float y) 
{
    return x * y * y;
}
\end{lstlisting}

\begin{figure}[htp]
\centering
\includegraphics[width=0.4\linewidth]{Screenshot 2024-01-30 223600.png}
\caption{\label{fig:text file}  }
\end{figure}

\begin{lstlisting}[language=C++, caption={Text file}]
//function to create a xyz file with the function f(x,y) = x*y^2
void createXYZFile(const char* filename, int resolution) 
{
    ofstream file(filename);

    if (!file.is_open()) 
    {
        cerr << "Unable to open file: " << filename << endl;
        return;
    }

   //define the area of the function
    float minX = -5.0f;
    float maxX = 5.0f;
    float minY = -5.0f;
    float maxY = 5.0f;

    //define the stepsize
    float stepX = (maxX - minX) / resolution;
    float stepY = (maxY - minY) / resolution;

    int numLines = 0;

   //loop through the area and calculate the function f(x,y) = x*y^2
    for (float x = minX; x <= maxX; x += stepX) 
    {
        for (float y = minY; y <= maxY; y += stepY) 
        {
            float z = f(x, y);
            file << x << " " << y << " " << z << endl;
            numLines++;
        }
    }

    file.close();

   
    ifstream readFile(filename);
    string fileContents;
    fileContents += to_string(numLines) + "\n";
    fileContents += string((istreambuf_iterator<char>(readFile)), 
    istreambuf_iterator<char>());
    readFile.close();

    ofstream writeFile(filename);
    writeFile << fileContents;
    writeFile.close();
}
\end{lstlisting}

\section{Discussion}
I am not that happy about my result but that is because I lack knowledge. Task 1 is the one that I am most happy about because I manage to have the color green when the graph is positive and red when its negative. I am not happy about that I couldn't make show more of the graph.
In task 2 I didn't get a spiral I only got the text file and a half a circle with color.
Task 3 I didn't get the graph of the function i choose, just the text file.

What I have learned is different ways to put color on. 

\section{GitHub Link}
\url{https://github.com/RebekkaOskarsen/Compulsory1_fall24.git}

\section{Resources}
ChatGPT, \url{https://chat.openai.com/}

\end{document}